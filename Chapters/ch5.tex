\chapter{SYSTEM REQUIREMENTS}

\hspace{0.5cm}The implementation of the proposed "Mind Matrix AI" system requires specific computational resources to handle the high-dimensional nature of fMRI data and the training of 3D Convolutional Neural Networks (CNNs). This chapter outlines the hardware and software specifications necessary for the development, training, and deployment phases of the project.

\section{Hardware Specifications}
\hspace{0.5cm}Given the computational intensity of processing volumetric brain data and performing sliding-window operations, the hardware selection focuses on multi-core processing and parallel computation capabilities. While the system is capable of running inference on standard CPUs, training the deep learning models efficiently requires a dedicated Graphics Processing Unit (GPU).

\vspace{0.5em}

Table \ref{tab:hardware_reqs} details the minimum requirements for basic operation and the recommended specifications for optimal training performance.

\vspace{1.5em}


\begin{table}[h]
	\centering
	\renewcommand{\arraystretch}{1.5} % Adds vertical space for readability
	% Defined with vertical lines |...|
	\begin{tabular}{|p{0.2\textwidth}|p{0.35\textwidth}|p{0.35\textwidth}|}
		\hline
		\textbf{Component} & \textbf{Minimum Configuration} & \textbf{Recommended Configuration} \\
		\hline
		\textbf{Processor (CPU)} & Quad-core (Intel Core i5 or AMD Ryzen 5) & Octa-core (Intel Core i7 or AMD Ryzen 7) or higher \\
		\hline
		\textbf{Memory (RAM)} & 16 GB DDR4 & 32 GB DDR4 or higher (Crucial for loading large .nii files) \\
		\hline
		\textbf{Graphics (GPU)} & Integrated Graphics (CPU-only training is slow) & NVIDIA RTX 3060 (12GB VRAM) or higher for CUDA acceleration \\
		\hline
		\textbf{Storage} & 100 GB SSD & 512 GB NVMe SSD (High I/O throughput for data loading) \\
		\hline
	\end{tabular}
	\caption{Hardware Requirements Summary}
	\label{tab:hardware_reqs}
\end{table}

\section{Software Environment}
\hspace{0.5cm}The project is built entirely on the Python ecosystem, leveraging its extensive libraries for scientific computing and deep learning. The software stack is designed to be cross-platform, compatible with Linux, Windows, and macOS environments.

The specific libraries and their version requirements are categorized in Table \ref{tab:software_reqs}.

\begin{table}[h]
	\centering
	\renewcommand{\arraystretch}{1.4}
	% Switched to p columns to ensure full width and wrapping for long notes
	\begin{tabular}{|p{0.25\textwidth}|p{0.25\textwidth}|p{0.4\textwidth}|}
		\hline
		\textbf{Category} & \textbf{Software/Library} & \textbf{Version / Note} \\
		\hline
		\textbf{Core Environment} & Python & 3.8 -- 3.10 \\
		& Operating System & Ubuntu 20.04+, Windows 10/11 \\
		\hline
		\textbf{Deep Learning}    & TensorFlow & $\ge$ 2.8.0 \\
		& Keras & Included in TensorFlow \\
		\hline
		\textbf{Neuroimaging}     & Nilearn & $\ge$ 0.9.0 (For statistical learning) \\
		& Nibabel & $\ge$ 3.2.0 (For .nii file I/O) \\
		\hline
		\textbf{Machine Learning} & Scikit-learn & $\ge$ 1.0.0 (For SVM/RF baselines) \\
		\hline
		\textbf{Data Processing}  & NumPy & $\ge$ 1.19.0 \\
		& Pandas & $\ge$ 1.3.0 \\
		& SciPy & $\ge$ 1.7.0 \\
		\hline
	\end{tabular}
	\caption{Software and Library Dependencies}
	\label{tab:software_reqs}
\end{table}