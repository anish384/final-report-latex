% Chapter 9: Conclusion
\chapter{CONCLUSION}

\hspace{0.5cm}This project successfully demonstrates that deep learning substantially advances automated emotion recognition from functional MRI data through automatic feature learning and hierarchical representation discovery. The developed CNN-based brain decoding system achieves 84.3\% accuracy on four-class emotion classification (happy, sad, angry, neutral), significantly exceeding classical machine learning baselines including Support Vector Machines (72.1\%), Random Forests (76.3\%), and Logistic Regression (68.4\%). The system's key innovation lies in its multi-atlas integration strategy, combining functional connectivity features from Harvard-Oxford, AAL, and Destrieux atlases to capture brain organization across multiple spatial scales, resulting in 2.9\% accuracy improvement over single-atlas approaches. 

Rigorous Leave-One-Subject-Out cross-validation demonstrates robust generalization to novel individuals with low variability, confirming genuine emotion-specific learning rather than subject-specific memorization. Feature importance analysis reveals neurobiologically plausible discriminative patterns involving limbic-prefrontal circuits, default mode networks, and attention systems, aligning with established emotion neuroscience literature. While limitations exist including modest sample size and validation on a single experimental paradigm, the demonstrated performance establishes deep learning as superior for brain decoding applications. 
